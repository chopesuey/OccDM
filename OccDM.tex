\documentclass[11pt,a4paper]{ivoa}
\input tthdefs
\input gitmeta

% Packages minimaux selon ivoatex conventions
% Note: listings, graphicx, hyperref sont déjà chargés par ivoa.cls
\usepackage{todonotes}

% Configuration listings pour exemples ADQL/SQL
\lstloadlanguages{SQL}
\lstset{flexiblecolumns=true,basicstyle=\ttfamily\small,
  showstringspaces=false}

\title{Stellar Occultation Data Model}

\ivoagroup{Data Models}

\author[https://orcid.org/0009-0009-8818-4561]{Tanguy Chope}

\editor{Tanguy Chope}
\editor{Damya Souami}
\editor{Yücel Kılıç}

\previousversion{This is the first public release}


\begin{document}

\begin{abstract}
This document defines a data model for stellar occultation observations 
by small Solar System bodies (asteroids, Trans-Neptunian Objects, Centaurs, 
Trojans, and comets). Stellar occultations occur when a Solar System body 
passes in front of a background star as seen from Earth, causing a 
characteristic drop in the star's apparent brightness. The resulting 
light curves encode high-precision information about the occulting body's 
size, shape, and any surrounding structures.

The model is designed as an extension to EPN-TAP 2.0 (Europlanet Table 
Access Protocol), following established IVOA patterns and leveraging 
existing standards including VO-DML, Coords, and Meas. The source 
characterization component (OccultedStar) is designed to be compatible 
with MANGO for epoch propagation support.

The architecture follows a strict 2-class design:
\begin{itemize}
\item \textbf{OccultationEvent} --- Primary class containing event records 
      with embedded star and body parameters
\item \textbf{OccultationFeature} --- Secondary class for auxiliary 
      detections (rings, atmosphere, satellites)
\end{itemize}

Individual chord observations are accessed via the DataLink protocol, 
consistent with the ObsCore pattern for accessing related data products.
\end{abstract}


\section*{Acknowledgments}
We acknowledge valuable input from the IVOA Data Model Working Group, 
particularly regarding VO-DML formalization and MANGO compatibility. 
Special thanks to the EPN-TAP community for guidance on planetary data 
standards and to CDS (Centre de Données astronomiques de Strasbourg) 
for technical review.

This work is supported by Observatoire de Paris, LIRA (Laboratoire 
d'Informatique et de Recherche en Astronomie), and the PADC (Paris 
Astronomical Data Centre).


\section*{Conformance-related definitions}

The words ``MUST'', ``SHALL'', ``SHOULD'', ``MAY'', ``RECOMMENDED'', and
``OPTIONAL'' (in upper or lower case) used in this document are to be
interpreted as described in IETF standard RFC2119 \citep{std:RFC2119}.

The \emph{Virtual Observatory (VO)} is a
general term for a collection of federated resources that can be used
to conduct astronomical research, education, and outreach.
The \href{https://www.ivoa.net}{International
Virtual Observatory Alliance (IVOA)} is a global
collaboration of separately funded projects to develop standards and
infrastructure that enable VO applications.

Conformance levels for implementations of this data model:
\begin{itemize}
\item \textbf{Level 1 (Minimal)}: Implementation of OccultationEvent 
      with mandatory parameters only. Service exposes data via TAP.
\item \textbf{Level 2 (Standard)}: Level 1 plus OccultationFeature and 
      DataLink endpoint for chord-level data access.
\item \textbf{Level 3 (Full)}: Level 2 plus MIVOT annotations for MANGO 
      compatibility and visualization product URLs.
\end{itemize}


\section{Introduction}

\subsection{Role within the VO Architecture}

\begin{figure}[th]
\centering
\includegraphics[width=0.9\textwidth]{role_diagram.pdf}
\caption{Architecture diagram for this document. OccDM (highlighted) 
is positioned in the Data Models area, building upon VODML, Meas, 
and Coords, with MANGO compatibility. It interfaces with TAP for 
queries and DataLink for chord data access.}
\label{fig:archdiag}
\end{figure}

Fig.~\ref{fig:archdiag} shows the role this document plays within the
IVOA architecture \citep{2021ivoa.spec.1101D}.

The model relates to other IVOA standards as follows:
\begin{itemize}
\item \textbf{Base layer (VO-DML)}: Provides the formal modeling 
      framework and primitive types \citep{2018ivoa.spec.0910L}.
\item \textbf{Measurement layer (Coords, Meas)}: Provides coordinate 
      systems and measurement types with errors 
      \citep{2022ivoa.spec.0611R,2022ivoa.spec.0611M}.
\item \textbf{Source layer (MANGO)}: Provides patterns for source 
      characterization (OccultedStar mapping).
\item \textbf{Discovery layer (EPN-TAP)}: Provides the query interface 
      and base parameter vocabulary \citep{2022ivoa.spec.0822E}.
\item \textbf{Access layer (DataLink)}: Provides access to chord-level 
      data products \citep{2023ivoa.spec.0609D}.
\item \textbf{Annotation layer (MIVOT)}: Enables annotation of VOTable 
      responses for client interpretation \citep{2023ivoa.spec.0620M}.
\end{itemize}


\subsection{Motivation}

Stellar occultations occur when a Solar System body passes in front 
of a background star as seen from Earth. The star's light is blocked, 
producing a characteristic light curve that encodes information about 
the occulting body's size, shape, and any surrounding structures 
(rings, atmosphere, satellites). These observations provide unique 
constraints on physical parameters that are difficult to obtain by 
other means:

\begin{itemize}
\item \textbf{Size and shape}: Kilometric precision, far exceeding 
      direct imaging capabilities for distant objects
\item \textbf{Ring systems}: Detection of structures a few kilometers 
      wide at distances of tens of AU
\item \textbf{Atmospheres}: Sensitivity to nanobar pressure levels, 
      orders of magnitude better than spectroscopy
\item \textbf{Satellites}: Detection of companions too faint or too 
      close for direct imaging
\item \textbf{Geometric albedo}: Combined with absolute magnitude, 
      provides albedo without assumptions about shape
\end{itemize}

Over the past decades, occultation programs have accumulated thousands 
of observations. These data have led to fundamental discoveries 
including: the rings of Chariklo \citep{2014Natur.508...72B}, the 
rings of Haumea \citep{2017Natur.550..219O}, and the bi-lobed shape 
of Arrokoth \citep{2019Sci...364.9771S}.

\textbf{The problem}: Currently, occultation data is scattered across 
individual publications, personal databases, and institutional archives 
with no common format. Researchers must contact individual groups and 
manually convert data formats to combine observations.

\textbf{The solution}: A standardized data model that enables 
interoperability across archives, federated queries through the 
Virtual Observatory infrastructure, and systematic analysis across 
multiple events and targets.


\subsection{Context and Scope}

This document defines a data model for stellar occultation observations 
by small Solar System bodies. The scope includes:
\begin{itemize}
\item Event-level metadata (timing, geometry, observer configuration)
\item Derived physical parameters (size, shape, albedo)
\item Secondary feature detections (rings, atmosphere, satellites)
\item Individual chord observations (via DataLink)
\item Occulted star characterization (compatible with MANGO)
\item Occulting body identification and orbital elements
\end{itemize}

\textbf{Out of scope}: Raw light curve data (FITS files), prediction 
ephemerides, real-time alert systems, lunar occultations, exoplanet 
transits.

The model is designed as an extension to EPN-TAP 2.0, the Europlanet 
Table Access Protocol for planetary science data. It follows the 
patterns established by ObsCore \citep{2017ivoa.spec.0509L} for 
observation discovery and DataLink for accessing related data products.


%%%%%%%%%%%%%%%%%%%%%%%%%%%%%%%%%%%%%%%%%%%%%%%%%%%%%%%%%%%%%
\section{Use Cases and Requirements}
%%%%%%%%%%%%%%%%%%%%%%%%%%%%%%%%%%%%%%%%%%%%%%%%%%%%%%%%%%%%%

This section presents the primary use cases that have motivated the 
development of this data model, followed by the derived requirements.

\subsection{Use Cases}

\subsubsection{UC1: Discovery of All Occultations for a Given Target}

\textbf{Actor}: Planetary scientist studying small Solar System objects

\textbf{Goal}: Find all recorded occultation events for a targeted SSO

\textbf{Scenario}: The scientist queries a TAP service with the target 
name or MPC designation. The service returns all events for that target, 
including derived parameters (diameter, albedo) and indicators for 
secondary features (rings, atmosphere).

\textbf{Derived requirements}: Unambiguous target identification 
(MPC designation), queryable via ADQL, support for name resolution.


\subsubsection{UC2: Access to Individual Chord Data}

\textbf{Actor}: Researcher performing shape modeling or re-analysis

\textbf{Goal}: Retrieve all chord observations (ingress/egress times, 
observer locations) for a specific event to perform an independent fit

\textbf{Scenario}: After discovering an event of interest, the 
researcher follows the DataLink URL to retrieve a VOTable containing 
all chord records. Each chord includes timing measurements with 
uncertainties, observer geodetic coordinates, and instrumentation 
metadata.

\textbf{Derived requirements}: DataLink endpoint, chord timing with 
errors, observer location, timing source (GPS/NTP).


\subsubsection{UC3: Survey of Ring-Bearing Small Bodies}

\textbf{Actor}: Researcher studying ring system formation and stability

\textbf{Goal}: Find all small bodies with confirmed or suspected ring 
detections

\textbf{Scenario}: Query for events where has\_rings = 'yes' or 
'suspected'. For positive detections, retrieve ring parameters 
(radius, width, optical depth) from the OccultationFeature class.

\textbf{Derived requirements}: Feature indicators in main class, 
separate feature class with detailed parameters.


\subsubsection{UC4: Temporal Evolution of Atmosphere/Rings}

\textbf{Actor}: Researcher monitoring atmospheric evolution 
(e.g., Pluto, Triton)

\textbf{Goal}: Track changes in atmospheric pressure or ring optical 
depth across multiple events over years/decades

\textbf{Scenario}: Query all events for a target, filter by feature 
type, compare derived parameters across epochs.

\textbf{Derived requirements}: Precise event timing (JD UTC), 
consistent parameter definitions with uncertainties, bibliographic 
references for traceability.


\subsubsection{UC5: Star Position Epoch Propagation}

\textbf{Actor}: Prediction service or occultation planner

\textbf{Goal}: Propagate the occulted star's position to the event 
epoch for shadow path calculation

\textbf{Scenario}: Retrieve star parameters (position, proper motion, 
parallax, errors) and propagate to event\_epoch using standard 
astrometric transformation. MANGO-compatible clients can perform 
this automatically.

\textbf{Derived requirements}: Complete astrometric parameters with 
correlated errors, MANGO compatibility for OccultedStar.


\subsubsection{UC6: Cross-Match with External Catalogs}

\textbf{Actor}: Data center or researcher combining occultation data 
with other sources

\textbf{Goal}: Link occultation-derived parameters with thermal 
observations, radar measurements, or spectroscopic data

\textbf{Scenario}: Use MPC designation as primary key to join with 
external databases (SBDB, Horizons, SsODNet).

\textbf{Derived requirements}: Standardized target identification, 
alternative designations array, target class vocabulary consistent 
with EPN-TAP.


\subsubsection{UC7: Visualization of Event Results}

\textbf{Actor}: Researcher or outreach coordinator

\textbf{Goal}: Access graphical representations of event results

\textbf{Scenario}: Retrieve URLs to pre-generated visualization 
products showing body limb with chord projections and terrestrial 
map of observer station distribution.

\textbf{Derived requirements}: URL fields for visualization products.


\subsection{Requirements}

Requirements are classified as MUST (essential), SHOULD (recommended), 
or MAY (optional). The reference to motivating use cases is indicated 
in brackets.

\subsubsection{General Requirements}

\textbf{[R1]} [MUST] The model SHALL be expressible in VO-DML and 
serializable to VO-DML/XML.

\textbf{[R2]} [MUST] The model SHALL reuse existing IVOA primitive 
types (ivoa:string, ivoa:real, ivoa:RealQuantity).

\textbf{[R3]} [MUST] The model SHALL import coords and meas models 
for coordinate and measurement types.

\textbf{[R4]} [MUST] The model SHALL be implementable as an EPN-TAP 
extension. [Stakeholder constraint]

\textbf{[R5]} [SHOULD] The model SHOULD provide MANGO-compatible 
source characterization for OccultedStar. [UC5]

\subsubsection{Application Requirements}

\textbf{[R6]} [MUST] Services implementing this model SHALL expose 
data via TAP with ADQL query support. [UC1, UC3]

\textbf{[R7]} [MUST] Services SHALL provide DataLink endpoints for 
chord-level data access. [UC2]

\textbf{[R8]} [SHOULD] Services SHOULD provide MIVOT annotations 
for MANGO-compatible clients. [UC5]

\textbf{[R9]} [MUST] Services SHALL register in the IVOA Registry 
with appropriate resource types. [UC6]

\textbf{[R10]} [MAY] Services MAY provide visualization product URLs. 
[UC7]

\subsubsection{Content Requirements}

\textbf{[R11]} [MUST] Event records SHALL include MPC packed 
designation for unambiguous body identification. [UC1, UC6]

\textbf{[R12]} [MUST] Event records SHALL include sufficient stellar 
parameters to enable epoch propagation. [UC5]

\textbf{[R13]} [MUST] Timing measurements SHALL be expressed in JD 
(UTC) with stated precision. [UC2, UC4]

\textbf{[R14]} [MUST] Derived parameters SHALL include associated 
uncertainties. [UC4]

\textbf{[R15]} [MUST] Chord data SHALL include observer geodetic 
coordinates and timing source. [UC2]

\textbf{[R16]} [MUST] All numerical parameters SHALL specify unit 
and number of decimal places. [All]


%%%%%%%%%%%%%%%%%%%%%%%%%%%%%%%%%%%%%%%%%%%%%%%%%%%%%%%%%%%%%
\section{Model: StellarOccultation}
%%%%%%%%%%%%%%%%%%%%%%%%%%%%%%%%%%%%%%%%%%%%%%%%%%%%%%%%%%%%%

\subsection{Model Identification}

\begin{table}[th]
\begin{tabular}{ll}
\sptablerule
\textbf{Property} & \textbf{Value} \\
\sptablerule
name & StellarOccultation \\
version & 1.0 \\
uri & \nolinkurl{ivo://ivoa.net/dm/StellarOccultation} \\
author & T. Chope (LIRA/Observatoire de Paris) \\
lastModified & 2026-01-16 \\
\sptablerule
\end{tabular}
\caption{Model identification}
\label{tab:modelid}
\end{table}


\subsection{Model Imports}

This model imports the following IVOA models:

\begin{table}[th]
\begin{tabular}{lll}
\sptablerule
\textbf{Model} & \textbf{Version} & \textbf{Usage} \\
\sptablerule
ivoa & 1.0 & Primitive types (string, real, boolean, RealQuantity, anyURI) \\
coords & 1.0 & LonLatPoint (observer geodetic position) \\
meas & 2.0 & Time, Position, ProperMotion, GenericMeasure (with errors) \\
\sptablerule
\end{tabular}
\caption{Model imports}
\label{tab:imports}
\end{table}


\subsection{Type Summary}

\textbf{ObjectTypes} (classes that may be persisted):
\begin{itemize}
\item \textbf{OccultationEvent} --- Primary class containing 
      occultation event records
\item \textbf{OccultationFeature} --- Secondary class for auxiliary 
      detections
\end{itemize}

\textbf{DataTypes} (embedded or accessed via DataLink):
\begin{itemize}
\item \textbf{OccultedStar} --- Background star parameters 
      (embedded in OccultationEvent)
\item \textbf{OccultingBody} --- Small body parameters 
      (embedded in OccultationEvent)
\item \textbf{ChordData} --- Individual chord observation 
      (accessed via DataLink, NOT persisted)
\end{itemize}

\textbf{Enumerations}:
\begin{itemize}
\item \textbf{TargetClass} --- asteroid $|$ TNO $|$ Centaur $|$ 
      dwarf\_planet $|$ comet $|$ satellite $|$ Trojan
\item \textbf{StatusFlag} --- yes $|$ no $|$ suspected
\item \textbf{DuplicityFlag} --- single $|$ binary $|$ 
      suspected\_binary $|$ multiple $|$ unknown
\item \textbf{ChordStatus} --- positive $|$ negative $|$ grazing
\item \textbf{TimingSource} --- GPS $|$ NTP $|$ manual $|$ unknown
\item \textbf{FeatureType} --- ring $|$ atmosphere $|$ satellite
\item \textbf{DetectionType} --- detection $|$ upper\_limit $|$ marginal
\end{itemize}


\subsection{Architecture Diagram}

\begin{figure}[th]
\centering
% Note: This diagram will be generated from Modelio + VO-DML module
% For now, use the hand-crafted SVG converted to PDF
%\includegraphics[width=\textwidth]{architecture_diagram.pdf}
\caption{VO-DML class diagram of the Stellar Occultation Data Model. 
Two ObjectTypes (green), three DataTypes (blue for embedded, cyan 
for DataLink-accessed), seven enumerations (yellow), model imports 
(purple), and MANGO compatibility annotations.}
\label{fig:vodml}
\end{figure}


%%%%%%%%%%%%%%%%%%%%%%%%%%%%%%%%%%%%%%%%%%%%%%%%%%%%%%%%%%%%%
\section{OccultationEvent}
%%%%%%%%%%%%%%%%%%%%%%%%%%%%%%%%%%%%%%%%%%%%%%%%%%%%%%%%%%%%%

The primary ObjectType representing a stellar occultation event. 
One instance represents exactly one event: a single small body 
occulting a single star at a specific epoch.

\subsection{Event Identification}

\subsubsection{event\_id}
\begin{bigdescription}
\item[vodml-id] OccultationEvent.event\_id
\item[type] ivoa:string
\item[multiplicity] 1 (mandatory)
\end{bigdescription}

Unique identifier for this occultation event within the service. 
The recommended format is \texttt{YYYYMMDD\_TARGETNAME} where 
TARGETNAME is the MPC packed designation (e.g., 
\texttt{20240610\_K03M12W} for Varda). This identifier serves as 
the primary key and is used for DataLink chord access queries.


\subsubsection{event\_epoch}
\begin{bigdescription}
\item[vodml-id] OccultationEvent.event\_epoch
\item[type] meas:Time
\item[multiplicity] 1 (mandatory)
\end{bigdescription}

The geocentric closest approach time in Julian Date (UTC). This is 
the nominal occultation time, typically corresponding to the midpoint 
of the central chord or the predicted closest approach if no central 
chord was observed. The meas:Time type bundles a coords:TimeStamp 
with measurement errors. Precision: 0.1 second (JD with 6 decimal 
places).


\subsubsection{shadow\_velocity}
\begin{bigdescription}
\item[vodml-id] OccultationEvent.shadow\_velocity
\item[type] meas:GenericMeasure
\item[multiplicity] 1 (mandatory)
\end{bigdescription}

The velocity of the shadow across the Earth's surface in km/s. 
This is the apparent velocity combining the body's orbital motion, 
Earth's rotation, and parallax effects. Negative values indicate 
retrograde motion. This parameter is critical for converting timing 
measurements to physical distances 
(chord\_length = shadow\_velocity $\times$ $\Delta$t). 
Unit: km/s, precision: 2 decimal places.


\subsubsection{shadow\_pa}
\begin{bigdescription}
\item[vodml-id] OccultationEvent.shadow\_pa
\item[type] ivoa:RealQuantity
\item[multiplicity] 0..1
\end{bigdescription}

Position angle of the shadow motion direction, measured East of 
North. Unit: degrees, precision: 1 decimal place.


\subsubsection{closest\_approach}
\begin{bigdescription}
\item[vodml-id] OccultationEvent.closest\_approach
\item[type] ivoa:RealQuantity
\item[multiplicity] 1 (mandatory)
\end{bigdescription}

The geocentric closest approach distance between the body center 
and the star, expressed as an angular separation. Unit: arcsec, 
precision: 3 decimal places.


\subsubsection{event\_duration}
\begin{bigdescription}
\item[vodml-id] OccultationEvent.event\_duration
\item[type] meas:GenericMeasure
\item[multiplicity] 0..1
\end{bigdescription}

Maximum duration of the occultation (for the central chord if 
available). Unit: seconds, precision: 1 decimal place.


\subsection{Chord Statistics}

\subsubsection{nb\_chords\_positive}
\begin{bigdescription}
\item[vodml-id] OccultationEvent.nb\_chords\_positive
\item[type] ivoa:nonnegativeInteger
\item[multiplicity] 1 (mandatory)
\end{bigdescription}

Number of positive chord detections (observers who recorded a 
clear occultation with both ingress and egress).


\subsubsection{nb\_chords\_negative}
\begin{bigdescription}
\item[vodml-id] OccultationEvent.nb\_chords\_negative
\item[type] ivoa:nonnegativeInteger
\item[multiplicity] 0..1
\end{bigdescription}

Number of negative observations (observers who did not detect 
the occultation despite adequate coverage).


\subsubsection{nb\_chords\_grazing}
\begin{bigdescription}
\item[vodml-id] OccultationEvent.nb\_chords\_grazing
\item[type] ivoa:nonnegativeInteger
\item[multiplicity] 0..1
\end{bigdescription}

Number of grazing chords (partial occultations near the limb edge).


\subsubsection{datalink\_url}
\begin{bigdescription}
\item[vodml-id] OccultationEvent.datalink\_url
\item[type] ivoa:anyURI
\item[multiplicity] 1 (mandatory)
\end{bigdescription}

URL to the DataLink service endpoint for accessing individual 
chord observations. Query pattern: 
\texttt{GET \{datalink\_url\}?ID=\{event\_id\}} returns a VOTable 
with ChordData structure. The service follows IVOA DataLink 1.1 
protocol.


\subsection{Derived Physical Parameters}

Parameters derived from the occultation fit. These represent the 
primary scientific output and require at least one positive chord.

\subsubsection{equiv\_diameter}
\begin{bigdescription}
\item[vodml-id] OccultationEvent.equiv\_diameter
\item[type] meas:GenericMeasure
\item[multiplicity] 0..1
\end{bigdescription}

The area-equivalent circular diameter: $D = 2\sqrt{A/\pi}$ where 
$A$ is the projected area from the ellipse fit. Unit: km, 
precision: 1 decimal place.


\subsubsection{ellipse\_a}
\begin{bigdescription}
\item[vodml-id] OccultationEvent.ellipse\_a
\item[type] meas:GenericMeasure
\item[multiplicity] 0..1
\end{bigdescription}

Semi-major axis of the best-fit ellipse to the limb points. 
Unit: km, precision: 1 decimal place.


\subsubsection{ellipse\_b}
\begin{bigdescription}
\item[vodml-id] OccultationEvent.ellipse\_b
\item[type] meas:GenericMeasure
\item[multiplicity] 0..1
\end{bigdescription}

Semi-minor axis of the best-fit ellipse. Unit: km, 
precision: 1 decimal place.


\subsubsection{ellipse\_pa}
\begin{bigdescription}
\item[vodml-id] OccultationEvent.ellipse\_pa
\item[type] ivoa:RealQuantity
\item[multiplicity] 0..1
\end{bigdescription}

Position angle of the ellipse major axis, measured East of North 
in the sky plane. Unit: degrees, precision: 1 decimal place.


\subsubsection{geometric\_albedo}
\begin{bigdescription}
\item[vodml-id] OccultationEvent.geometric\_albedo
\item[type] meas:GenericMeasure
\item[multiplicity] 0..1
\end{bigdescription}

Geometric albedo derived from the occultation diameter combined 
with the absolute magnitude. Calculated using: 
$p_v = (1329/D)^2 \times 10^{-0.4 \times H}$ where $D$ is 
equiv\_diameter in km and $H$ is the absolute magnitude. 
Dimensionless, precision: 2 decimal places.


\subsubsection{oblateness}
\begin{bigdescription}
\item[vodml-id] OccultationEvent.oblateness
\item[type] ivoa:real
\item[multiplicity] 0..1
\end{bigdescription}

Oblateness defined as $(a-b)/a$ where $a$ and $b$ are the 
semi-major and semi-minor axes. Dimensionless, precision: 
2 decimal places.


\subsection{Secondary Feature Indicators}

Boolean-like flags indicating the presence of secondary structures. 
When set to 'yes', corresponding detailed parameters are available 
in the OccultationFeature class.

\subsubsection{has\_rings}
\begin{bigdescription}
\item[vodml-id] OccultationEvent.has\_rings
\item[type] StatusFlag
\item[multiplicity] 0..1
\end{bigdescription}

Indicates ring detection. Values: 'yes' (confirmed), 'no' 
(ruled out with upper limits), 'suspected' (marginal detection).


\subsubsection{has\_atmosphere}
\begin{bigdescription}
\item[vodml-id] OccultationEvent.has\_atmosphere
\item[type] StatusFlag
\item[multiplicity] 0..1
\end{bigdescription}

Indicates atmosphere detection. Values: 'yes' (confirmed), 
'no' (ruled out), 'suspected' (marginal detection).


\subsubsection{has\_satellite}
\begin{bigdescription}
\item[vodml-id] OccultationEvent.has\_satellite
\item[type] StatusFlag
\item[multiplicity] 0..1
\end{bigdescription}

Indicates satellite detection via secondary occultation event.


\subsection{Metadata}

\subsubsection{bib\_reference}
\begin{bigdescription}
\item[vodml-id] OccultationEvent.bib\_reference
\item[type] ivoa:string
\item[multiplicity] 0..1
\end{bigdescription}

Bibcode or DOI of the primary publication describing this event 
(e.g., \texttt{2020A\&A...643A.125S} for Varda).


\subsubsection{timing\_source}
\begin{bigdescription}
\item[vodml-id] OccultationEvent.timing\_source
\item[type] TimingSource
\item[multiplicity] 1 (mandatory)
\end{bigdescription}

Timestamp synchronization method: 'GPS' (best), 'NTP', 'manual', 
'unknown'. Critical for assessing timing accuracy.


\subsection{Compositions}

OccultationEvent contains two embedded DataTypes:

\subsubsection{star : OccultedStar [1]}

Composition of the occulted star characterization. See Section~5.

\subsubsection{target : OccultingBody [1]}

Composition of the occulting body characterization. See Section~6.


%%%%%%%%%%%%%%%%%%%%%%%%%%%%%%%%%%%%%%%%%%%%%%%%%%%%%%%%%%%%%
\section{OccultedStar}
%%%%%%%%%%%%%%%%%%%%%%%%%%%%%%%%%%%%%%%%%%%%%%%%%%%%%%%%%%%%%

DataType for background star characterization. Embedded in 
OccultationEvent via composition. MANGO-compatible for epoch 
propagation support.

\subsection{Star Identification}

\subsubsection{star\_id}
\begin{bigdescription}
\item[vodml-id] OccultedStar.star\_id
\item[type] ivoa:string
\item[multiplicity] 1 (mandatory)
\end{bigdescription}

Unique identifier in the reference catalog, typically Gaia DR3 
source\_id (e.g., \texttt{4367203805493754752}).


\subsubsection{star\_catalogue}
\begin{bigdescription}
\item[vodml-id] OccultedStar.star\_catalogue
\item[type] ivoa:string
\item[multiplicity] 1 (mandatory)
\end{bigdescription}

Reference catalog name. Recommended values: 'GaiaDR3', 'GaiaDR2', 
'UCAC4', 'Tycho-2'.


\subsection{Astrometry}

\subsubsection{position}
\begin{bigdescription}
\item[vodml-id] OccultedStar.position
\item[type] meas:Position
\item[multiplicity] 1 (mandatory)
\end{bigdescription}

Star position at catalog reference epoch (typically J2016.0 for 
Gaia DR3). The meas:Position type bundles coords:LonLatPoint 
with errors and correlations. Precision: 6 decimal places 
for degrees ($\sim$3.6 mas).


\subsubsection{proper\_motion}
\begin{bigdescription}
\item[vodml-id] OccultedStar.proper\_motion
\item[type] meas:ProperMotion
\item[multiplicity] 0..1
\end{bigdescription}

Proper motion in RA and Dec. Unit: mas/yr, precision: 3 decimal 
places.


\subsubsection{star\_ra\_error, star\_dec\_error}
\begin{bigdescription}
\item[vodml-id] OccultedStar.star\_ra\_error / star\_dec\_error
\item[type] ivoa:RealQuantity
\item[multiplicity] 0..1
\end{bigdescription}

Position uncertainties. Unit: mas, precision: 2 decimal places.


\subsubsection{parallax}
\begin{bigdescription}
\item[vodml-id] OccultedStar.parallax
\item[type] ivoa:RealQuantity
\item[multiplicity] 0..1
\end{bigdescription}

Trigonometric parallax. Unit: mas, precision: 3 decimal places.


\subsection{Photometry}

\subsubsection{star\_mag\_G, star\_mag\_RP, star\_mag\_BP}
\begin{bigdescription}
\item[vodml-id] OccultedStar.star\_mag\_G / RP / BP
\item[type] ivoa:real
\item[multiplicity] G: 1, RP/BP: 0..1
\end{bigdescription}

Gaia photometry bands. star\_mag\_G is mandatory as it determines 
the SNR requirements. Precision: 2 decimal places.


\subsubsection{star\_mag\_J, star\_mag\_H, star\_mag\_K}
\begin{bigdescription}
\item[vodml-id] OccultedStar.star\_mag\_J / H / K
\item[type] ivoa:real
\item[multiplicity] 0..1
\end{bigdescription}

2MASS infrared magnitudes. Precision: 2 decimal places.


\subsection{Quality Indicators}

\subsubsection{star\_ruwe}
\begin{bigdescription}
\item[vodml-id] OccultedStar.star\_ruwe
\item[type] ivoa:real
\item[multiplicity] 0..1
\end{bigdescription}

Gaia Renormalized Unit Weight Error. RUWE $>$ 1.4 may indicate 
binarity or astrometric problems. Precision: 2 decimal places.


\subsubsection{star\_duplicity}
\begin{bigdescription}
\item[vodml-id] OccultedStar.star\_duplicity
\item[type] DuplicityFlag
\item[multiplicity] 0..1
\end{bigdescription}

Stellar multiplicity status. Values: 'single', 'binary', 
'suspected\_binary', 'multiple', 'unknown'.


\subsubsection{angular\_diameter}
\begin{bigdescription}
\item[vodml-id] OccultedStar.angular\_diameter
\item[type] meas:GenericMeasure
\item[multiplicity] 0..1
\end{bigdescription}

Estimated stellar angular diameter (from color relations). 
Unit: mas, precision: 2 decimal places.


\subsubsection{spectral\_type}
\begin{bigdescription}
\item[vodml-id] OccultedStar.spectral\_type
\item[type] ivoa:string
\item[multiplicity] 0..1
\end{bigdescription}

MK spectral classification if known (e.g., 'K2III').


%%%%%%%%%%%%%%%%%%%%%%%%%%%%%%%%%%%%%%%%%%%%%%%%%%%%%%%%%%%%%
\section{OccultingBody}
%%%%%%%%%%%%%%%%%%%%%%%%%%%%%%%%%%%%%%%%%%%%%%%%%%%%%%%%%%%%%

DataType for occulting body characterization. Embedded in 
OccultationEvent via composition.

\subsection{Target Identification}

\subsubsection{target\_name}
\begin{bigdescription}
\item[vodml-id] OccultingBody.target\_name
\item[type] ivoa:string
\item[multiplicity] 1 (mandatory)
\end{bigdescription}

MPC packed designation (e.g., \texttt{K03M12W} for (174567) Varda). 
This is the primary key for cross-matching. See 
\url{https://minorplanetcenter.net/iau/info/PackedDes.html}.


\subsubsection{target\_name\_readable}
\begin{bigdescription}
\item[vodml-id] OccultingBody.target\_name\_readable
\item[type] ivoa:string
\item[multiplicity] 0..1
\end{bigdescription}

Human-readable designation (e.g., \texttt{(174567) Varda}).


\subsubsection{target\_class}
\begin{bigdescription}
\item[vodml-id] OccultingBody.target\_class
\item[type] TargetClass
\item[multiplicity] 1 (mandatory)
\end{bigdescription}

Body classification. Values: 'asteroid', 'TNO', 'Centaur', 
'dwarf\_planet', 'comet', 'satellite', 'Trojan'. Based on 
EPN-TAP vocabulary.


\subsubsection{dynamical\_class}
\begin{bigdescription}
\item[vodml-id] OccultingBody.dynamical\_class
\item[type] ivoa:string
\item[multiplicity] 0..1
\end{bigdescription}

Detailed dynamical classification (e.g., \texttt{classical-hot}, 
\texttt{resonant-3:2}). This value is frozen at event\_epoch.


\subsection{Physical Properties}

\subsubsection{target\_geocentric\_distance}
\begin{bigdescription}
\item[vodml-id] OccultingBody.target\_geocentric\_distance
\item[type] ivoa:RealQuantity
\item[multiplicity] 1 (mandatory)
\end{bigdescription}

Distance from Earth at event epoch. Unit: AU, precision: 
5 decimal places.


\subsubsection{ephemeris\_source}
\begin{bigdescription}
\item[vodml-id] OccultingBody.ephemeris\_source
\item[type] ivoa:string
\item[multiplicity] 0..1
\end{bigdescription}

Ephemeris used for predictions (e.g., \texttt{NIMAv13}, 
\texttt{JPL\#45}).


\subsection{Orbital Elements}

\subsubsection{semi\_major\_axis}
\begin{bigdescription}
\item[vodml-id] OccultingBody.semi\_major\_axis
\item[type] ivoa:RealQuantity
\item[multiplicity] 0..1
\end{bigdescription}

Semi-major axis of the heliocentric orbit. Unit: AU, 
precision: 2 decimal places.


\subsubsection{eccentricity}
\begin{bigdescription}
\item[vodml-id] OccultingBody.eccentricity
\item[type] ivoa:real
\item[multiplicity] 0..1
\end{bigdescription}

Orbital eccentricity. Dimensionless, precision: 3 decimal places.


\subsubsection{inclination}
\begin{bigdescription}
\item[vodml-id] OccultingBody.inclination
\item[type] ivoa:RealQuantity
\item[multiplicity] 0..1
\end{bigdescription}

Orbital inclination to the ecliptic. Unit: degrees, 
precision: 1 decimal place.


%%%%%%%%%%%%%%%%%%%%%%%%%%%%%%%%%%%%%%%%%%%%%%%%%%%%%%%%%%%%%
\section{ChordData (via DataLink)}
%%%%%%%%%%%%%%%%%%%%%%%%%%%%%%%%%%%%%%%%%%%%%%%%%%%%%%%%%%%%%

DataType for individual chord observations. Accessed via DataLink 
protocol, NOT persisted as a database table. Each chord represents 
one observer's recording of the occultation event.

\subsection{Chord Identification}

\subsubsection{chord\_id}
\begin{bigdescription}
\item[vodml-id] ChordData.chord\_id
\item[type] ivoa:string
\item[multiplicity] 1 (mandatory)
\end{bigdescription}

Unique identifier for this chord within the event. Recommended 
format: \texttt{event\_id\_NNN} (sequential numbering).


\subsubsection{parent\_event\_id}
\begin{bigdescription}
\item[vodml-id] ChordData.parent\_event\_id
\item[type] ivoa:string
\item[multiplicity] 1 (mandatory)
\end{bigdescription}

Reference to the parent OccultationEvent.event\_id.


\subsection{Timing Measurements}

\subsubsection{ingress\_time}
\begin{bigdescription}
\item[vodml-id] ChordData.ingress\_time
\item[type] meas:Time
\item[multiplicity] 0..1
\end{bigdescription}

Time when the star disappears (start of occultation). Unit: JD (UTC), 
precision: 0.001 second. May be absent for negative chords.


\subsubsection{egress\_time}
\begin{bigdescription}
\item[vodml-id] ChordData.egress\_time
\item[type] meas:Time
\item[multiplicity] 0..1
\end{bigdescription}

Time when the star reappears (end of occultation). Unit: JD (UTC), 
precision: 0.001 second.


\subsubsection{chord\_length}
\begin{bigdescription}
\item[vodml-id] ChordData.chord\_length
\item[type] meas:GenericMeasure
\item[multiplicity] 0..1
\end{bigdescription}

Length of the chord calculated from timing and shadow velocity. 
Unit: km, precision: 1 decimal place.


\subsubsection{chord\_status}
\begin{bigdescription}
\item[vodml-id] ChordData.chord\_status
\item[type] ChordStatus
\item[multiplicity] 1 (mandatory)
\end{bigdescription}

Quality flag: 'positive' (full occultation), 'negative' 
(no detection), 'grazing' (partial).


\subsection{Observer Location}

\subsubsection{observer\_location}
\begin{bigdescription}
\item[vodml-id] ChordData.observer\_location
\item[type] coords:LonLatPoint
\item[multiplicity] 1 (mandatory)
\end{bigdescription}

Geodetic position of the observing station. Longitude East positive, 
latitude North positive. Precision: 5 decimal places ($\sim$1 m).


\subsubsection{observer\_altitude}
\begin{bigdescription}
\item[vodml-id] ChordData.observer\_altitude
\item[type] ivoa:RealQuantity
\item[multiplicity] 0..1
\end{bigdescription}

Altitude above WGS84 ellipsoid. Unit: meters, precision: 1 decimal 
place.


\subsection{Instrumentation}

\subsubsection{telescope\_aperture}
\begin{bigdescription}
\item[vodml-id] ChordData.telescope\_aperture
\item[type] ivoa:RealQuantity
\item[multiplicity] 0..1
\end{bigdescription}

Telescope aperture diameter. Unit: meters, precision: 2 decimal 
places.


\subsubsection{timing\_source}
\begin{bigdescription}
\item[vodml-id] ChordData.timing\_source
\item[type] TimingSource
\item[multiplicity] 1 (mandatory)
\end{bigdescription}

Timestamp synchronization method for this specific station.


\subsubsection{snr}
\begin{bigdescription}
\item[vodml-id] ChordData.snr
\item[type] ivoa:real
\item[multiplicity] 0..1
\end{bigdescription}

Signal-to-noise ratio of the detection. Dimensionless.


%%%%%%%%%%%%%%%%%%%%%%%%%%%%%%%%%%%%%%%%%%%%%%%%%%%%%%%%%%%%%
\section{OccultationFeature}
%%%%%%%%%%%%%%%%%%%%%%%%%%%%%%%%%%%%%%%%%%%%%%%%%%%%%%%%%%%%%

ObjectType for auxiliary detections (rings, atmosphere, satellites). 
Related to OccultationEvent via aggregation (one event may have 
multiple features).

\subsection{Feature Identification}

\subsubsection{feature\_id}
\begin{bigdescription}
\item[vodml-id] OccultationFeature.feature\_id
\item[type] ivoa:string
\item[multiplicity] 1 (mandatory)
\end{bigdescription}

Unique identifier for this feature. Recommended format: 
\texttt{event\_id\_type\_NNN}.


\subsubsection{parent\_event\_id}
\begin{bigdescription}
\item[vodml-id] OccultationFeature.parent\_event\_id
\item[type] ivoa:string
\item[multiplicity] 1 (mandatory)
\end{bigdescription}

Reference to OccultationEvent.event\_id. This establishes the 
aggregation relationship.


\subsubsection{feature\_type}
\begin{bigdescription}
\item[vodml-id] OccultationFeature.feature\_type
\item[type] FeatureType
\item[multiplicity] 1 (mandatory)
\end{bigdescription}

Discriminator: 'ring', 'atmosphere', 'satellite'. Determines 
which conditional parameters apply.


\subsubsection{detection\_type}
\begin{bigdescription}
\item[vodml-id] OccultationFeature.detection\_type
\item[type] DetectionType
\item[multiplicity] 1 (mandatory)
\end{bigdescription}

Quality flag: 'detection' (confirmed), 'upper\_limit' 
(non-detection with constraint), 'marginal' (tentative).


\subsection{Ring Parameters (feature\_type = 'ring')}

\subsubsection{ring\_name}
\begin{bigdescription}
\item[vodml-id] OccultationFeature.ring\_name
\item[type] ivoa:string
\item[multiplicity] 0..1
\end{bigdescription}

Ring designation (e.g., 'C1R', 'Q2R' for Chariklo/Quaoar rings).


\subsubsection{ring\_radius}
\begin{bigdescription}
\item[vodml-id] OccultationFeature.ring\_radius
\item[type] meas:GenericMeasure
\item[multiplicity] 0..1
\end{bigdescription}

Mean orbital radius from body center. Unit: km, precision: 
1 decimal place.


\subsubsection{ring\_width}
\begin{bigdescription}
\item[vodml-id] OccultationFeature.ring\_width
\item[type] meas:GenericMeasure
\item[multiplicity] 0..1
\end{bigdescription}

Radial width of the ring. Unit: km, precision: 1 decimal place.


\subsubsection{ring\_optical\_depth}
\begin{bigdescription}
\item[vodml-id] OccultationFeature.ring\_optical\_depth
\item[type] meas:GenericMeasure
\item[multiplicity] 0..1
\end{bigdescription}

Normal optical depth ($\tau$). Dimensionless, precision: 
2 decimal places.


\subsection{Atmosphere Parameters (feature\_type = 'atmosphere')}

\subsubsection{half\_light\_radius}
\begin{bigdescription}
\item[vodml-id] OccultationFeature.half\_light\_radius
\item[type] meas:GenericMeasure
\item[multiplicity] 0..1
\end{bigdescription}

Radius at 50\% flux level. Unit: km, precision: 1 decimal place.


\subsubsection{surface\_pressure}
\begin{bigdescription}
\item[vodml-id] OccultationFeature.surface\_pressure
\item[type] meas:GenericMeasure
\item[multiplicity] 0..1
\end{bigdescription}

Estimated surface pressure. Unit: $\mu$bar, precision: 
2 decimal places.


\subsubsection{scale\_height}
\begin{bigdescription}
\item[vodml-id] OccultationFeature.scale\_height
\item[type] meas:GenericMeasure
\item[multiplicity] 0..1
\end{bigdescription}

Atmospheric scale height. Unit: km, precision: 1 decimal place.


%%%%%%%%%%%%%%%%%%%%%%%%%%%%%%%%%%%%%%%%%%%%%%%%%%%%%%%%%%%%%
\section{Open Issues and Future Work}
%%%%%%%%%%%%%%%%%%%%%%%%%%%%%%%%%%%%%%%%%%%%%%%%%%%%%%%%%%%%%

\subsection{Dynamical Class Evolution}

Some objects have dynamical classifications that evolve due to 
gravitational perturbations. Example: (99942) Apophis transitions 
from ``Aten'' to ``Apollo'' class after its 2029 Earth encounter.

\textbf{Current solution}: The dynamical\_class field is frozen at 
event\_epoch. Future versions may include a time-dependent 
classification service reference.

\subsection{Binary Star Treatment}

Binary stars produce complex light curves with potential for 
multiple events. The current model provides star\_duplicity as 
a warning flag. Future versions may extend OccultedStar to 
handle resolved binary parameters.

\subsection{Contact Binary Bodies}

How to report on contact binaries (e.g., Arrokoth)? The current 
ellipse parameterization may not adequately describe bi-lobed 
shapes. Future work: Investigation of shape model parameters.

\subsection{MPC Designation Resolution}

Integration with MPC name resolver services for automatic 
designation cross-matching. Reference: 
\url{https://minorplanetcenter.net/web_service}.


\appendix

%%%%%%%%%%%%%%%%%%%%%%%%%%%%%%%%%%%%%%%%%%%%%%%%%%%%%%%%%%%%%
\section{Enumeration Values}
%%%%%%%%%%%%%%%%%%%%%%%%%%%%%%%%%%%%%%%%%%%%%%%%%%%%%%%%%%%%%

\subsection{TargetClass}
\begin{lstlisting}
asteroid | TNO | Centaur | dwarf_planet | comet | satellite | Trojan
\end{lstlisting}
Based on EPN-TAP target\_class vocabulary.

\subsection{StatusFlag}
\begin{lstlisting}
yes | no | suspected
\end{lstlisting}
For has\_rings, has\_atmosphere, has\_satellite indicators.

\subsection{DuplicityFlag}
\begin{lstlisting}
single | binary | suspected_binary | multiple | unknown
\end{lstlisting}
Stellar multiplicity classification.

\subsection{ChordStatus}
\begin{lstlisting}
positive | negative | grazing
\end{lstlisting}
positive = full occultation; negative = no detection; grazing = partial.

\subsection{TimingSource}
\begin{lstlisting}
GPS | NTP | manual | unknown
\end{lstlisting}
Timestamp synchronization method. GPS preferred.

\subsection{FeatureType}
\begin{lstlisting}
ring | atmosphere | satellite
\end{lstlisting}
Discriminator for OccultationFeature.

\subsection{DetectionType}
\begin{lstlisting}
detection | upper_limit | marginal
\end{lstlisting}
Quality of feature detection.


%%%%%%%%%%%%%%%%%%%%%%%%%%%%%%%%%%%%%%%%%%%%%%%%%%%%%%%%%%%%%
\section{ADQL Query Examples}
%%%%%%%%%%%%%%%%%%%%%%%%%%%%%%%%%%%%%%%%%%%%%%%%%%%%%%%%%%%%%

\subsection{Find all occultations for a target}

\begin{lstlisting}[language=SQL]
SELECT event_id, event_epoch, equiv_diameter, geometric_albedo
FROM occ_event
WHERE target_name = 'K03M12W'
ORDER BY event_epoch
\end{lstlisting}

\subsection{Find ring-bearing objects}

\begin{lstlisting}[language=SQL]
SELECT DISTINCT e.target_name_readable, f.ring_radius, f.ring_width
FROM occ_event e
JOIN occ_feature f ON e.event_id = f.parent_event_id
WHERE e.has_rings = 'yes' AND f.feature_type = 'ring'
\end{lstlisting}

\subsection{Atmospheric pressure evolution}

\begin{lstlisting}[language=SQL]
SELECT e.event_epoch, f.surface_pressure
FROM occ_event e
JOIN occ_feature f ON e.event_id = f.parent_event_id
WHERE e.target_name = 'J34P00I'  -- Pluto
  AND f.feature_type = 'atmosphere'
ORDER BY e.event_epoch
\end{lstlisting}


%%%%%%%%%%%%%%%%%%%%%%%%%%%%%%%%%%%%%%%%%%%%%%%%%%%%%%%%%%%%%
\section{Changes from Previous Versions}
%%%%%%%%%%%%%%%%%%%%%%%%%%%%%%%%%%%%%%%%%%%%%%%%%%%%%%%%%%%%%

No previous versions yet.

% When updating, use subsections like:
% \subsection{Changes from WD-1.0}
% \begin{itemize}
% \item Added xyz parameter
% \item Clarified abc requirement
% \end{itemize}


\bibliography{ivoatex/ivoabib,ivoatex/docrepo,localrefs}


\end{document}